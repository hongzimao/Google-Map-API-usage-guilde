\documentclass [12pt] {article}
\usepackage[top=1in, bottom=0.9in, left=1in, right=1in]{geometry}
\usepackage{graphicx}
\usepackage{setspace}
\usepackage{hyperref}
\usepackage{listings}
\usepackage{courier}
\usepackage[export]{adjustbox}
\usepackage[english]{babel}
\begin {document}
\begin{spacing}{1.5}
\noindent
\large
\begin {center}
\textbf{Set up Google Map API v2 on Mac with Eclipse}\\
\end {center}
~\\
\normalsize
\noindent
\url{https://developers.google.com/maps/documentation/android/start}\\
\\
\textbf{1. Install Google Play Service SDK }\\
This can be done in Eclipse$\rightarrow$Window$\rightarrow$Android SDK manager\\
~\\
\textbf{2. Associate Google Play Service Library}\\
Right click on project$\rightarrow$properties$\rightarrow$android$\rightarrow$add$\rightarrow$choose Google Play Service Library\\
~\\
\textbf{3. Add meta-data in Manifest}\\
Add
\lstset{basicstyle=\footnotesize\ttfamily,breaklines=true}
\lstset{framextopmargin=50pt}
 \begin{lstlisting}
 <meta-data
� � android:name="com.google.android.gms.version"
� � android:value="@integer/google_play_services_version" />
 \end{lstlisting}
right after $<$application$>$\\
~\\
\textbf{4. Get SHA1}\\
on terminal, type in 
\lstset{basicstyle=\footnotesize\ttfamily,breaklines=true}
\lstset{framextopmargin=50pt}
 \begin{lstlisting}
keytool -list -v -keystore ~/.android/debug.keystore -alias androiddebugkey -storepass android -keypass android
 \end{lstlisting}
 copy SHA1\\
 ~\\
 \textbf{5. Create object in Google}\\
 Go to \url{https://console.developers.google.com/project}\\
 Create a new project\\
 ~\\
 \textbf{6. Get API key}\\
In left selection panel, API. Enable Google Map API v2\\
In credentials, create new key\\
Choose Android key, type in SHAI;project package name\\
Copy the API key\\
~\\
\textbf{7. Add API key in Manifest}\\
in androidManifest.xml, add
\lstset{basicstyle=\footnotesize\ttfamily,breaklines=true}
\lstset{framextopmargin=50pt}
 \begin{lstlisting}
 <meta-data
� � android:name="com.google.android.maps.v2.API_KEY"
� � android:value="API_KEY"/>
 \end{lstlisting}
 right before $</$application$>$
 \\
 ~\\
 \textbf{8. Add internet permission}\\
 in androidManifest.xml, add 
 \lstset{basicstyle=\footnotesize\ttfamily,breaklines=true}
\lstset{framextopmargin=50pt}
 \begin{lstlisting}
 <uses-permission android:name="android.permission.INTERNET"/>
	<uses-permission android:name="android.permission.ACCESS_NETWORK_STATE"/>
	<uses-permission android:name="android.permission.WRITE_EXTERNAL_STORAGE"/>
	<uses-permission android:name="com.google.android.providers.gsf.permission.READ_GSERVICES"/>
	
	<uses-feature
        android:glEsVersion="0x00020000"
        android:required="true"/>
\end{lstlisting}
 right before $</$manifest$>$
 
\end {spacing}
\end {document}